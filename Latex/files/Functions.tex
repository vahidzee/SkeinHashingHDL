\chapter{توابع و ساختارها}

\section{ \textbf{ساختارها}}

\subsection{\lr{sph-skein-big-context}}
\label{subsec:sph-skein-big-context}
این ساختار مورد نظر برای ذخیره و استفاده از هش است (‌ شامل مقادیری از هش قبلی و مقادیر جدید محاسبه شده ). \\ این ساختار شامل یک آرایه‌ی ۶۴ بیتی از کاراکترهاست که به منظور تراز کردن انواع هش استفاده می‌گردد و  هشت عدد ۶۴ بیتی  که برای ذخیره‌ی ۵۱۲ بیت هش  استفاده می‌شوند  و هم‌چنین شامل دو عدد با نام‌های \lr{ptr, bcount} است که این دو عدد به طور معمول برابر ۰ هستند که همانند \lr{nonce} در پیاده‌سازی وریلاگ آن است.



\subsection{\lr{IV512}}
\label{subsec:IV512}
این ساختار شامل مقادیر اولیه‌ی هش است. یک عدد ۵۱۲ بیتی را برای خوانا بودن در مبنای ۱۶ و در ۸ بلاک ۱۶ بیتی نگاه می‌دارد. این مقدار در برنامه‌ به زبان وریلاگ همان \lr{midstate} است.
\subsection{\lr{UBI-BIG}}
\label{subsec:UBI-BIG}
این تابع  (  در مدل‌طلایی به صورت \lr{define} تعریف شده )  طبق الگوریتم \lr{skein} در ابتدا وظیفه‌ی ۵۱۲ بیتی کردن ورودی در بافر را بر عهده دارد، در ادامه تمامی ۷۲ مرحله‌ی هش الگوریتم \lr{skein} را که در مقدمه شرح داده شده است را اجرا می‌کند.
\\
روند اجرای این تابع بدین صورت است که در ابتدا مقادیر ۸ بیتی در بافر را با استفاده از \lr{Encoder} به مقادیر ۶۴ بیتی تبدیل می‌کند، سپس دو مقدار $t_0 , t_1$
را که همان $ tweak $ ها هستند را با استفاده از ورودی‌ها به‌دست می‌آورد و سپس با استفاده از تابع \hyperref[subsec:TFBIG-INIT]{\lr{TFBIG-INIT}} مقادیر جدیدی از داده‌های قبلی به دست می‌آورد،‌ سپس ۱۸ بار توابع \hyperref[subsec:TFBIG-4e]{\lr{TFBIG-4e}} و \hyperref[subsec:TFBIG-4o]{\lr{TFBIG-4o}} صدا می‌شوند که در هر کدام از‌ این تابع‌ها ۴ بار تابع درهم‌سازی توضیح داده شده در مقدمه صدا می‌شوند.


\subsection{\lr{TFBIG-4e, TFBIG-4o}}
\label{subsec:TFBIG-4e}
\label{subsec:TFBIG-4o}

این تابع برای کدگذاری $ P_0 $ تا $ P_7 $ طراحی شده‌است. همان‌طور که پیش‌تر توضیح داده شده است، ۷۲ بار تابع درهم‌سازی صدا می‌شود،‌ و هر ۸ سلسله از این ۷۲ مرحله یکسان است، هم‌چنین در هر ۸ سری ۴ بار با یک کلید و ۴ بار دیگر با یک کلید دیگر اجرا می‌شود،‌ که به همین دلیل این توابع  هر کدام برای آن ۴ باری استفاده می‌شود که در مرحله‌ای زوج یا فرد قرار داریم.
\\
این تابع یک ورودی  \lr{s} دارد. تابع  \lr{TFBIG-ADDKEY } با ‌$ p_0 $ تا  $ p_7 $ و $ h $ و $ t $و ‌$ s $ به ترتیب به عنوان  $ w_0 $ تا $ w_7 $و $ k $و ‌$ t $ و ‌$ s $ صدا شده‌است. ‌$ h $ و $ t $ برای \lr{concat} و ساختن کلید در تابع \lr{TFBIG-ADDKEY} استفاده شده‌اند.
سپس  \lr{TFBIG-MIX8 }چهار بار برای ترتیب‌های متفاوتی از ‌$ p_0 $ تا $ p_7 $ اعداد متفاوت به عنوان $ rc $ صدا شده‌است. ترتیب صدا شدن $ p_0 $ تا $ p_7 $ برای تعداد بلاک ۸ به صورت جدول‌های زیر است،‌ که برای هر ‌‌راند از ۰ تا ۳، بر حسب راند قبل ترتیب‌ها چهار عدد جابه‌جا شده‌اند. و تفاوت حالت‌های زوج و فرد در اعداد استفاده شده است.
\begin{center}
	\includegraphics[width=10cm]{images/table_mix.png}
	\includegraphics[width= 10cm]{images/Mix2.png}
\end{center}



\subsection{\lr{TFBIG-INIT}}
\label{subsec:TFBIG-INIT}
این تابع با ورودی‌های $ t_0 $ تا $ t_2 $ و $h_0 $ تا $ h_8 $ مقادیر زیر را محاسبه می‌کند :
$$
k_8 = C \oplus k_0 \oplus k_1 \oplus ... \oplus k_7
$$

$$
	t_2 = t_1 \oplus t_0
$$
 که مقدار ثابت $ C $ به آن جهت در فرمول وجود دارد که از ۰ نبودن تمامی بیت‌ها اطمینان حاصل شود.
\section{ توابع}

\subsection{\lr{sph-skein512-init}}
\label{subsec:sph-skein512-init}
این تابع مسئولیت مقداردهی اولیه‌ی ساختار هش را بر عهده دارد، که برای آن تابع \hyperref[subsec:skein-big-init]{\lr{skein-big-init}} را با ورودی‌ اولیه‌ی \lr{IV512} اجرا می‌کند.
\subsection{\lr{skein-big-init}}
\label{subsec:skein-big-init}

این تابع دو ورودی می‌پذیرد که یکی از آن‌ها آدرس یک ساختار هش است و دیگری مقدار اولیه، که مقادیر متناظر ساختار داده شده را برابر مقادیر اولیه قرار می‌دهد.
که مقدار اولیه در حالت ۵۱۲ بیتی در ساختار \lr{IV512} ذخیره شده ‌است.

\subsection{\lr{sph-skein512}}
\label{subsec:sph-skein512}
این تابع، مقدار هش محاسبه شده تا این لحظه را از بین می‌برد و 